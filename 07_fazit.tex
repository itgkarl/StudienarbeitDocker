\chapter{Fazit}
\label{sec:fazit}
Docker ist noch ein sehr junges Projekt das großes Potential in sich birgt. 
Docker Container ermöglichen eine leichtgewichtige, portable Virtualisierung. Die Entwicklung, das Testen und das Bereitstellen von Anwendungen werden durch die schnelle Startzeit, das erleichterte Verwalten von Abhängigkeiten und die geringe Größe vereinfacht. Die Anwendungsbereiche sind vielseitig und weit gestreut. Blickt man etwas über den Tellerrand hinaus merkt man schnell, dass Docker seine Berechtigung nicht nur im Cloud Computing findet sondern in vielen Bereichen darüber hinaus sinnvoll eingesetzt werden kann.
Neben den Anwendungsszenarien die in dieser Arbeit beleuchtet wurden, sind noch viele weitere denkbar. Beispielsweise ein Anwendungsrepository für Unternehmen oder eine sichere Sandbox für Entwickler.
Docker wird von einer schnell wachsenden Community begleitet die immer neue Anwendungsfälle für Docker Präsentieren. Eine Auswahl dieser Communityprojekte findet man auf der Website von Docker. \cite{docker_docker_2014} Eine weitere Zusammenstellung von interessanten Use Cases bietet die Seite docker.com \cite{docker_use_2014}
Getrieben von dieser Community ist zu erwarten, dass Docker weiter wachsen und an Beliebtheit gewinnen wird.
Viele Unternehmen und Projekte wie zum Beispiel das Jenkins Projekt setzen Docker bereits produktiv ein obwohl zum jetzigen Zeitpunkt wegen der rapiden Weiterentwicklung von einem produktiven Einsatz abgeraten wird.